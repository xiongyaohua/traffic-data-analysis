\chapter{贝叶斯参数估计}

根据上一章节所学的贝叶斯概率相关知识,可以引出贝叶斯参数估计。
贝叶斯参数估计从参数的先验知识和样本出发。将参数θ看成一个未知的随机变量,通过对第$i$类样本$D_i$的观察,使概率密度分布$P(D_i|θ)$转化为后验概率$P(θ|D_i)$,再求贝叶斯估计。

\section{贝叶斯线性回归}
贝叶斯线性回归主要由两部分组成:
\begin{itemize}
    \item 求解后验分布;
    \item 依据后验分布以及样本$D$,进行预测。
\end{itemize}
贝叶斯线性回归的模型与普通线性回归一致:
\begin{equation}
 y=ax+b+\epsilon
\end{equation}

\subsection{求解后验}
首先假设先验概率$P(H)$为均匀分布,即$f(a,b)=1$
\begin{equation}
 P(D/H)=P(D_1...D_n|H)=P(D_1|H)P(D_2|H)...P(D_n|H)
\end{equation}
\begin{equation}
 P(D_1|θ)=P(X_1,Y_1|θ)
\end{equation}
\begin{equation}
 y=ax+b+\epsilon
\end{equation}
残差
\begin{equation}
r=y-(ax+b)
\end{equation}
假设$r$~$N(0,\sigma^2)$,即服从正态分布
则
\begin{equation}
  P(D|θ)=P(X,Y|θ)=\frac{1}{\sqrt{2π}\sigma}e^\frac{-(y-(ax+b))^2}{2\sigma^2}
\end{equation}
假设:1.先验概率=1
      2.误差正态分布
\begin{equation}
f(a,b)=\quad \prod_{i=1}^n \frac{1}{\sqrt{2π}\sigma}e^\frac{-(r_i)^2}{2\sigma^2} \quad*P(θ)
\end{equation}
\begin{equation}
logf(a,b)=\sum_{i=1}^n -\frac{1}{2}\frac{r_i^2}{\sigma^2} \quad
\end{equation} 
\begin{equation}
 \underset{a,b} {argmax }{\sum_{i=1}^n }-\frac{(r_i)^2}{\sigma^2}
\end{equation}

正则化
\begin{equation}
 \underset{a,b} {argmin }{\sum_{i=1}^n }\frac{(r_i)^2}{\sigma^2}+\lambda\sum\theta^2
\end{equation}

来自不同分布的点
\begin{equation}
 \underset{a,b} {argmin }{\sum_{i=1}^n }-\frac{(r_i)^2}{\sigma_i^2}
\end{equation}

|          | 通用性 | 性能 |
|---------|---------|---------|
| 线性回归   | 好      | 差 |
| 贝叶斯回归 | 差      | 好 |

\subsection{预测}

\begin{equation}
 P(y|x)=\int P(y|\theta,x)P(\theta|D)d\theta
\end{equation}
