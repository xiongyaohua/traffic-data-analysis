%\documentclass[14pt,aspectratio=169]{beamer}
\documentclass[14pt]{beamer}
\usepackage{ctex}
\usepackage{bm}
\usepackage{color, colortbl}
\definecolor{HRed}{rgb}{1,.2,.2}
%\usepackage{xeCJK} % important! Without this Chinese fonts won't show
%\usepackage{xeCJKfntef}
\setsansfont{Noto Sans CJK SC Light}
\setCJKsansfont{Noto Sans CJK SC Light}
\setCJKmainfont{Noto Serif CJK SC Light}

\usefonttheme[onlymath]{serif} % formulars in serif font
\usefonttheme{professionalfonts} % 防止公式间距异常,参见https://www.zhihu.com/question/55492768
\parskip=10pt

%\newcommand{\mat}[1]{\bm{#1}}
\newcommand{\mat}[1]{\bm{#1}}
\renewcommand{\vec}[1]{\bm{#1}}
\DeclareMathOperator*{\argmin}{arg\,min}

\let\emph\relax % there's no \RedeclareTextFontCommand
\DeclareTextFontCommand{\emph}{\bfseries}
%%%%%%%%%%%%%%% 正文开始 %%%%%%%%%%%%%%%%%%%%%%%

\title{第3讲:线性回归}
\subtitle{分析数据从$y=ax+b$开始}
\author{熊耀华}
\institute{交通工程系}

\begin{document}

\begin{frame}
    \titlepage
\end{frame}

\section{什么是线性回归?}

\begin{frame}
    \frametitle{回归问题}

    回归(Regression)是一类数据处理的理论工具,用来寻找连续变量之间的函数规律。

    \begin{description}
        \item[变量] 一个可以变化的数字,描述自然或社会的某种现象、属性。
        \item[连续] 可以在数轴某个区间上任意取值。
    \end{description}
\end{frame}

\begin{frame}
    \frametitle{回归问题}
    以出行需求预测问题为例,对某个交通小区,令$x$表示人口、$y$表示交通出行量,我们假设两者之间存在函数关系
        \[ y = f(x) \]

    核心问题是,如何确定函数$f$。
\end{frame}

\begin{frame}
    \frametitle{线性假设}
    函数$f$理论上可以是任何形式,但那样问题过于复杂。为了方便求解,我们假设$f$是一个线性函数
    \[y=ax+b\]

    此时核心问题变成,如何确定参数$a$和$b$,也称为线性回归(Linear Regression)。
\end{frame}

\begin{frame}
    \frametitle{确定参数$a$和$b$}
    假设我们随机调查了该城市中两个小区的出行情况,结果如下
    \begin{table}
        \begin{tabular}{l c c}
            小区编号 & 人口$x$ & 出行量$y$ \\
            \hline\hline
            1   & 1000  & 20 \\
            2   & 1500  & 35 \\
            
        \end{tabular}
    \end{table}

    根据这些数据可以确定参数$a$和$b$。
\end{frame}

\begin{frame}
    \frametitle{确定参数$a$和$b$}
    根据两组调查数据,我们列出未知数$a$、$b$的方程组
    \begin{align*}
        1000a+b&=20\\
        1500a+b&=35
    \end{align*}

    求解可得$a=0.03$、$b=-10$
\end{frame}

\begin{frame}
    \frametitle{确定参数$a$和$b$}
    前页中的方程组可以整理成矩阵形式
    \begin{align*}
        \begin{bmatrix}
            1000 & 1\\
            1500 & 1
        \end{bmatrix}\cdot
        \begin{bmatrix}
            a \\
            b
        \end{bmatrix}=
        \begin{bmatrix}
            20\\
            35
        \end{bmatrix}
    \end{align*}
    或者
    \[\mat{A}\cdot \vec{\theta}=\vec{b}\]

    参数$\vec{\theta}$有唯一解的前提条件是$\mat{A}$可逆,此时
    \[ \vec{\theta}= \mat{A}^{-1}\cdot\vec{b}\]
\end{frame}

\begin{frame}
    \frametitle{新数据带来问题}
    假如我们调查了一个新的小区
    \begin{table}
        \begin{tabular}{l c c}
            小区编号 & 人口$x$ & 出行量$y$ \\
            \hline\hline
            1   & 1000  & 20 \\
            2   & 1500  & 35 \\
            \rowcolor{HRed} 3 & 1200 & 30      
        \end{tabular}
    \end{table}

    新数据给拟合带来困难。
\end{frame}

\begin{frame}
    \frametitle{新数据带来问题}
    此时方程变为
    \begin{align*}
        \begin{bmatrix}
            1000 & 1\\
            1500 & 1\\
            1200 & 1\\
        \end{bmatrix}\cdot
        \begin{bmatrix}
            a \\
            b
        \end{bmatrix}=
        \begin{bmatrix}
            20\\
            35\\
            30
        \end{bmatrix}
    \end{align*}

    $\vec{\theta}=
    \begin{bmatrix}
        a & b
    \end{bmatrix}^T$没有解,因为
    \[
        \mat{A}=    
        \begin{bmatrix}
            1000 & 1\\
            1500 & 1\\
            1200 & 1\\
        \end{bmatrix}
        \]
    中约束的数量大于未知数的数量。
    \end{frame}

    \begin{frame}
        \frametitle{寻找最佳拟合}
        对于3组数据我们无法找到完美拟合的参数$\vec{\theta}$,我们退求其次
        寻找最佳拟合$\vec{\theta}^*$。
        
        如何定义最佳?残差的平方和最小。
        \begin{description}
            \item[残差] 观测值和预测值之差。观测值为$\vec{b}$,对给定参数$\vec{\theta}$预测值为$\mat{A}\cdot\vec{\theta}$,因此残差为$\vec{e}=\vec{b}-\mat{A}\cdot\vec{\theta}$ 
            \item[平方和] 求和可以得到总体误差;取平方可以避免正负抵消。 
        \end{description}
    \end{frame}
    \begin{frame}
        \frametitle{平方和的内积形式}
        平方和可以缩写成向量内积形式
        \begin{align*}
            \sum_{i=1}^n e_i^2&= e_1e_1+e^2e^2+\cdots\\
                    &=\begin{bmatrix}
                        e_1 & e_2 & \cdots
                    \end{bmatrix}\cdot
                    \begin{bmatrix}
                        e_1\\
                        e_2\\
                        \vdots
                    \end{bmatrix}\\
                    &=\vec{e}^T\cdot\vec{e}
        \end{align*}
    \end{frame}

    \begin{frame}
        \frametitle{寻找最佳拟合}
        基于以上分析,直线拟合问题转化为最优化问题,$\vec{\theta}$取什么值时$\vec{e}^T\vec{e}$最小。
        \[ \min_{\vec{\theta}} \vec{e}^T\vec{e}\]
        这个最优值表示为
        \[\vec{\theta}^*=\argmin \vec{e}^T\vec{e} \]

        这个问题又叫平方和最小问题(Least Square),也翻译成\emph{最小二乘}问题。
    \end{frame}
    
    \begin{frame}
        \frametitle{最优化}
    
        
    
    \end{frame}
\end{document}